\documentclass[titlepage]{article}
\usepackage{graphicx} % Required for inserting images

\usepackage[T1]{fontenc}
\usepackage{tgbonum}

\def\companyName{shmap}

\begin{document}

\begin{titlepage}
    \begin{center}
        \vspace*{1.5cm}

        {\fontfamily{phv}\selectfont
            {\Huge \companyName}
        }

        \vspace{0.5cm}
            {\large Convenient, united frontend for connecting\\
        stores, brands and customers}
            
        \vspace{1.25cm}

        \textbf{Project Group E1} \\
        {\normalsize Viktor Jernberg (vi6832je-s@student.lu.se)\\Thomas Dellwik (thomas.dellwik00gmail.com)\\Nils Reberg (ni7877re-s@student.lu.se)\\Nicolas Jaua Otero (ni1407ja-s@student.lu.se)\\Marina Fridh-Cardoso (ma0448fr-s@student.lu.se)\\Erik Nicander (er0811ni-s@student.lu.se)}

        \vspace{1.25cm}

        \includegraphics[width=0.45\textwidth]{logo_gradient.png}
       
        \vspace{1.5cm}

        {\Large Lund, Sweden}

        \vspace{0.5cm}
        
        {\Large 2024}\\
            
    \end{center}
\end{titlepage}

\section{Executive Summary}
Our product provides a new and effective way for in-store shoppers to navigate smoothly through any storefront. A mobile application with a streamlined and effective indoor navigation system that guides users through stores, allowing large shopping lists to be completed in quicker, more comfortable ways for the user. 

For customers, i.e large retail stores, we provide valuable data on user activity within the stores, allowing them to evaluate under/over-utilized areas of their locations while enabling effective advertisement of select products based on shopper data and interests. 

Our team, containing experienced algorithm developers and with previous experience in the retail industry, are suited to create a product that understands both what stores and store-goers are wanting for in the current system.

\section{Management team}
Our team consists of six computer science students who attend Lund University. We are currently in our third year of our studies and have learned a great deal about software and programming. Even though we study together, the members of the team have varying experiences that are relevant for this project.

The members are: Thomas Dellwik, Marina Fridh-Cardoso, Nicolas Jaua Otero, Viktor Jernberg, Erik Nicander and Nils Reberg. The different relevant experiences that the members have will help with early development and serves as a great base for us to get started. For example: Nicolas has previously worked with startups and has designed apps, Erik is great at design concepts and has done graphic design, and Viktor has worked a management role at a grocery store which is relevant because grocery stores are the target customer.

The main concern about competence missing from the team  is legal education and possibly leadership experience. If we could have a person with legal knowledge in our team, our team would essentially be complete as a startup.

\section{Business Model}
The main customer segment of our product are retail and grocery stores, and the contingent of shoppers who appreciate assistance, 
guidance or finding deals using cloud services. 
Many customers of stores have difficulties in finding where items in aisles are.
Unlike current solutions, which give minimal or no information to the consumer on where to go for various items,
our solution aims to instead give the consumer the ability to find items with a mobile solution,
letting them obtain information of product positions without having to actively seek them out.
In addition, the staff requirement of providing information to store customers would be greatly relaxed.

Usage and improvement of our product are tightly interconnected, as by user data such as store customer positions over time 
and user-provided map complementary data, our product improves. 
Unlike other solutions, our model can also combine the data from different model consumers, 
facilitating a high potential for growth.

By initially releasing our product for a lower price than a projected market price for a similar solution,
a market position as a leader in the product space would be obtained.
After obtaining a larger market share, a revenue stream by advertisements, 
licensing fees and selling the user data would be lucrative.

\begin{figure}[h]
    \centering
    \includegraphics[width=1\textwidth]{lean model canvas.png}
    \caption{Lean model canvas describing \companyName's business model}
    \label{fig:enter-label}
\end{figure}

\section{Market Analysis}
Our product would be aimed at in-store shoppers, people who want help getting around different stores and getting the best possible deals for what they are searching for. There are multiple current alternatives that cover part of what we aim for. For example, many supermarket stores have their own apps that show weekly deals, searching for products, among other things. Other apps like Tiendeo offer the possibility of seeing the catalog of discounts in nearby stores in your area. And the stores themselves usually display their own discounts and offer physical catalogs with their offers. 

Though this helps when searching for deals, it doesn’t help customers find their way within the store, where they would rely on looking around, checking the signs or asking for help from the store personnel. Pointr offers the possibility of indoor positioning, but it hasn’t reached the Swedish market, and it’s limited to countries like the US and the UK. 

Our product has the ability of combining the side of navigation within the store, optimizing routes according to items or shopping lists, plus promoting the user with deals and offering the option to compare product prices in different stores from the comfort of a single app.

For our initial market we would aim to partner with some individual stores in limited locations to show both users and customers, like supermarket chains, the possibilities our app offers. We would start with a smaller city, like Lund, and promote the app in front of the stores we’d be partnered with. We would aim our product mainly to bigger families, who want to optimize the time they spend in stores and the amount of money they spend on products.

\section{Implementation plan}

\section{Risk assessment}

A risk assessment of the entirety of this project concludes that significant risks stem from two sources: lack in interest from users and businesses, and risks associated with the inexperience of the team.

A failure to attract enough attention from and involvement of users in our product would make it significantly more difficult to attract investors as well as attract advertising companies, therefore directly affecting our main stream of revenue. Lacking interest from businesses will also make it difficult expand the usability of our product, in turn making it less desirable for users.

In order to mitigate these risks, MOM-tests and prototyping will help the team ensure that there is a market for the product, and that the problem-solution fit is good. Exclusive offers for partner businesses and users at the initial stages may help attract more of both categories. Prototyping at different stages in development, both for user side application and business side maintenance, to further solidify the solution is a good fit will help create a successful product. Offering help with setup and integration into existing warehouse management systems can help ease businesses who are concerned with the amount of work required, for set-up and maintenance alike. 

Inexperience in the team give rise to risks involving poor or insufficient planning, and an unrealistic schedule and budget. The plan may lack in granularity in turn leading to inaccurate estimates. Some important tasks may be overlooked or forgotten, or things may be planned out in a suboptimal way. Resources may be wrongly distributed causing a delay in some part of the project while an other part is stalled waiting for the delayed part.

By carefully applying well tested and known methods in creating a system anatomy and implementation plan, we hope to mitigate some of the risk. Methods like planning poker to estimate effort on different activities, and trying to give a little more slack than believed necessary some extra room for error should be created. Regular check-ins with the team to see the progress will allow potential problems to be discovered earlier so that action may be taken to minimize consequences. If possible a mentor or a review by someone with more experience in software related business could also be helpful when finalizing the implementation plan.

A complete risk assessment can be found in the appendix \ref{risktable}.

\section{Appendix}
\subsection{Contribution statement}
Thomas Dellwik: Part of group brainstorming and discussions when forming our business idea and MOM-test questions. Responsible for writing the Business model section in the first draft of the report. Spent around 5 hours total. \\
\\
Marina Fridh-Cardoso: Participated in discussions in regards to formulating and refining our business idea. Strengths and weaknesses, how to create value for both customers (retailers) and users. Active in planning and executing MOM-test. Together with team members filled out the Lean Canvas. Formalia and structure of report draft. Created prototypes. Responsible for risk assessment section. \\
\\
Nicolas Jaua Otero: Active part in discussions and planning of business idea and strategy. Contributed to MOM-test planning and execution. Wrote the Market analysis section of the report. About 5 hrs work.\\
\\
Viktor Jernberg: Took part in brainstorming, refining and planning business idea and business strategy, following the Lean Canvas Model. Active part of planning and executing the MOM test. Responsible for writing first draft of Management team section. About 5 hours.\\
\\
Erik Nicander: Participated in group discussions and planning, attended seminars and group planning meetings. Took an active role in developing our business idea and in fleshing out the business strategy and filled in the Lean Canvas. Approximately 5 hours of work in total.\\
\\
Nils Reberg: Took on a leading role in coming up with and refining the business idea and strategy, and completing Lean Canvas.  Involved in planning and execution of MOM-test. Author of the Executive summary section. Total time about 5 hrs spent.\\

\subsection{Risk Assessment Table} \label{risktable}

\begin{center}
\begin{tabular}{|p{3cm}|p{2cm}|p{2cm}|p{2cm}|p{2cm}|p{2cm}|}
\hline
 Specification & Probability & Consequences & Total Risk Estimation & Mitigation & Monitoring \\
 \hline
 \end{tabular}
\end{center}

\begin{center}
\begin{tabular}{|p{3cm}|p{2cm}|p{2cm}|p{2cm}|p{2cm}|p{2cm}|}
 \hline
 \multicolumn{6}{|c|}{User Level} \\
 \hline
 Low involvement of users. Not interested in new tech/solution, do not consider the "problem" substantial enough to want a solution for it. Not interested in learning how to use it. & medium & high & medium & MOM-test. Prototyping to ensure simple, easy to use UI, good fit for the problem. Exclusive offers to attract customers. & Prototype to get feedback \\  
 \hline
 Low interest from businesses. Perceive gains as small compared to work required. Not willing to learn new technology. & high & high & high & Prioritize simple and quick maintenance on business-end. Integration into their WMS. Prototyping. On site support to set up. Offer training to staff. & Prototype to get feedback \\  
 \hline
 \end{tabular}
\end{center}

 \begin{center}
 \begin{tabular}{|p{3cm}|p{2cm}|p{2cm}|p{2cm}|p{2cm}|p{2cm}|}
 \hline
 \multicolumn{6}{|c|}{Requirement gathering} \\
 \hline
 Requirements not deliverable: cannot get precise enough indoor location services for mapping & low & medium & low & Early testing. Adjust plan? Plan for alternative solution. & Research before starting implementation, alternative solution? \\
 \hline
 Misunderstanding or incomplete requirements. Misunderstood MOM-test results, and prospective users expect something other than what we have in mind. Missing important requirement for users to adopt our solution. Unspoken requirements: Do not want to create yet another account, do not wish to see a bunch of ads (which are part of our generating revenue). & low & medium & low & Prototyping, feedback including ads. & Prototype to get feedback \\
 \hline
 \end{tabular}
 \end{center}

 \begin{center}
 \begin{tabular}{|p{3cm}|p{2cm}|p{2cm}|p{2cm}|p{2cm}|p{2cm}|}
 \hline
 \multicolumn{6}{|c|}{Planning} \\
 \hline
 Unrealistic schedule and budget, unrealistic goals. Inexperienced team leads to overestimation of our skills and an unrealistic timeplan. & medium & medium & medium & Leave extra room. Planning poker to estimate effort for different activities. If possible aqcuire a business-mentor to help and give feedback on how realistic our plan is. & Regular check-ins with timeplan \\
 \hline
 Poor planning. Not detailed enough/ insufficient granularity. Missed or overlooked steps. & medium & medium & medium & SPM-tools to try and develop a more detailed plan to minimize risk of missing something important. Planning poker: if large variation in estimated effort, or large effort for a given task, perhaps a sign it can (should) be broken down further? & Regular check-ins with timeplan \\
 \hline
 \end{tabular}
 \end{center}



\begin{center}
 \begin{tabular}{|p{3cm}|p{2cm}|p{2cm}|p{2cm}|p{2cm}|p{2cm}|}
 \hline
 \multicolumn{6}{|c|}{Analysis} \\
 \hline
 Faulty cost-benefit analysis: planned development of a minor feature turns out to be much more expensive than initially expected & medium & low & low & Periodic revisions of the plan and re-prioritizing items. Using a fine granularity when planning to clarify what needs to get done. & Regular check-ins with timeplan \\
 \hline
 \end{tabular}
 \end{center}

 \begin{center}
 \begin{tabular}{|p{3cm}|p{2cm}|p{2cm}|p{2cm}|p{2cm}|p{2cm}|}
 \hline
 \multicolumn{6}{|c|}{Design} \\
 \hline
 Lack of integrity in design elements, bad or difficult to use UI & low & low & low & Perform prototyping to get user feedback on UI & Prototype tp get feedback \\
 \hline
 Lack of proper architectural design and plan & low & low & low & Separate activity to plan proper architecture. Create structure documents, UML-diagrams, and design meetings with team to ensure everyone is on board. Document structure. & Periodic code reviews \\
 \hline
 \end{tabular}
 \end{center}

\begin{center}
 \begin{tabular}{|p{3cm}|p{2cm}|p{2cm}|p{2cm}|p{2cm}|p{2cm}|}
 \hline
 \multicolumn{6}{|c|}{Implementation} \\
 \hline
 Risk for bugs & high & low & low & Test first, clean branch with only working code to ensure no broken code contaminates the working space. Internal code reviews. & Periodic code reviews, continous testing \\
 \hline
 Lack of proper documentation and comments, difficult to understand for other team-members & medium & low & low & Plan and create clear guidelines on documentation for produced code. Periodic internal code reviews. & Periodic code reviews \\
 \hline
 Lack of necessary skills & low & medium & low & Clear resource planning to ensure the best fit between resource and task. Team standup-meetings to quickly discover if there is a problem, so that resources may be redistributed. & Team check-ins \\
 \hline
 \end{tabular}
 \end{center}

 \begin{center}
 \begin{tabular}{|p{3cm}|p{2cm}|p{2cm}|p{2cm}|p{2cm}|p{2cm}|}
 \hline
 \multicolumn{6}{|c|}{Maintenance} \\
 \hline
 Lack of proper documentation & medium & low & low & Clear guidelines on documentation. Internal code reviews regularly. & Periodic code reviews, continous testing \\
 \hline
 Too complex for businesses to maintain: too involved, too much work, too difficult for staff & medium & medium & medium & Prototyping business side UI, get feedback on documentation, offer ongoing support. Already in planning stage plan for integration into existing WMS to minimize necessary effort from businesses to maintain. & Pre-release feedback \\
 \hline
 Political or legal changes that have a negative impact & low & low & low & Alternative ways of generating revenue if laws on gathering user data tightens for example & Monitor politics and news in relevant sector \\
 \hline
 \end{tabular}
 \end{center}


\end{document}
